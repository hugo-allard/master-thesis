\documentclass{beamer}

\beamertemplatenavigationsymbolsempty

\colorlet{red}{red!60!black}
\colorlet{green}{green!60!black}
\xdefinecolor{state}{RGB}{255, 244, 141}

\usepackage[francais]{babel}
\usepackage[utf8]{inputenc}
\usepackage[T1]{fontenc}

\usepackage{amsmath}
\usepackage{amssymb}
\usepackage{amsthm}
\usepackage{mathabx}
\usepackage{mathrsfs}

\usepackage{tikz}
\usetikzlibrary{automata, positioning}
\usetikzlibrary{decorations.pathmorphing}
\usetikzlibrary{decorations.markings}

\tikzset{onslide/.code args={<#1>#2}{%
		\only<#1>{\pgfkeysalso{#2}}
	}}

\title[Algorithmes pour les transducteurs finis]{Implémentation d'algorithmes de décision pour la fonctionnalité et la sous-séquentialité de transducteurs finis}
\author{Hugo \textsc{Allard}}
\institute[Umons]{Université de Mons}
\date{07/09/2015}

\usetheme{Madrid}

\begin{document}
	
	\begin{frame}
		\titlepage
	\end{frame}
	
	\AtBeginSection[]
	{
		\begin{frame}
			\frametitle{Sommaire}
			\tableofcontents[currentsection, hideothersubsections]
		\end{frame} 
	}
	
	\section{Automates finis}
		\begin{frame}
			\frametitle{Automates finis}
\framesubtitle{}

\begin{itemize}
	\item Accepte des mots sur un alphabet $\Sigma$
	\item Reconnaît un langage $\subseteq \Sigma^*$
\end{itemize}

\begin{block}{Définition}
	Un automate fini sur l'alphabet $\Sigma$ est un quintuplet $\mathscr{A} = (\Sigma,Q,I,F,\delta)$ où
	
	\begin{itemize}
		\item $\Sigma$ est l'alphabet d'entrée,
		\item $Q$ est l'ensemble fini des états,
		\item $I$ est l'ensemble des états initiaux,
		\item $F \subseteq Q$ est l'ensemble des états finaux,
		\item $\delta \subseteq Q \times \Sigma \times Q$ est la relation de transition.
	\end{itemize}
\end{block}

\centering
$L(\mathscr{A}) = \{w \in \Sigma^* \mid \exists$ un chemin réussi pour $w \}$

		\end{frame}
		
		\begin{frame}
			\frametitle{Automates finis}
\framesubtitle{}

Non-déterministe (NFA)
	\begin{figure}
	\begin{minipage}[b]{0.5\linewidth}

		\centering
		\begin{tabular}[b]{r||c|c|c}
			 & $a$ & $b$ & $\varepsilon$ \\
			\hline
			\hline
			$\to q_0$ & $\{q_0\}$ & $\{q_1,q_2\}$ & $\emptyset$ \\
			$*q_1$ & $\{q_1\}$ & $\emptyset$ & $\emptyset$ \\
			$q_2$ & $\{q_1\}$ & $\{q_0\}$ & $\{q_0\}$ \\
    			\end{tabular}
		\captionof*{table}{Table de transitions}
	\end{minipage}%
	\begin{minipage}[b]{0.5\linewidth}

		\centering
		\begin{tikzpicture}[%
					>=stealth,
					shorten >=1pt,
					node distance=3cm,
					on grid,
					auto,
					state/.append style={minimum size=2em},
					thick
					]

			\node[state] (A)		     {$q_0$};
			\node[state,accepting] (B) [below right of=A] {$q_1$};
			\node[state] (C) [below left of=A] {$q_2$};
				
			\path[->]
					+(-0.7,0.7) edge (A)
					(A)         edge             	 node {$b$} 		 (B)
					(A)         edge [bend left]    node {$b$} 		 (C)
					(A) 	   edge [loop above] node {$a$}		  ()
					(B)	   edge [loop above] node {$a$}		  ()
					(C)	   edge [bend left]	 node {$\varepsilon,b$} (A)
					(C)	   edge 		 node {$a$} 		  (B);
								
		\end{tikzpicture}
		\captionof*{figure}{Diagramme de transitions}

	\end{minipage}
				
	\caption{Automate fini asynchrone}
	\label{NFA}
\end{figure}

Déterministe (DFA)
	\begin{figure}
	\centering
	\begin{tikzpicture}[%
		>=stealth,
		shorten >=1pt,
		node distance=3cm,
		on grid,
		auto,
		state/.append style={fill=state, minimum size=2em},
		thick
		]
		
		\node[state,initial,initial text={}] (A)		     {$q_0$};
		\node[state] (B) [right of=A] {$q_1$};
		\node[state,accepting] (C) [right of=B] {$q_2$};
		\node[state,accepting] (D) [right of=C] {$q_3$};
		
		\path[->]
		(A)         edge [bend left]    node {$b$} 		 (B)
		(A) 	   edge [loop above] node {$a$}		  ()
		(B)	   edge [bend left] node {$a$}		  (A)
		(B)	   edge 		 node {$b$} 		  (C)
		(C)	   edge [loop above] node {$b$}		  ()
		(C)	   edge [bend left]   node {$a$} 		  (D)
		(D)	   edge [bend left]   node {$b$} 		  (C)
		(D)	   edge [loop above] node {$a$}		  ();
		
	\end{tikzpicture}
\end{figure}

		\end{frame}
		
		\begin{frame}
			\frametitle{Automates finis}
\framesubtitle{}

\begin{itemize}
	\item Les automates sont clos pour les opérations booléennes
	\item La plupart des problèmes de décisions sont décidables
	\item Pour chaque NFA, il existe un DFA équivalent
	\item Les automates caractérisent les \emph{langages rationnels}
\end{itemize}

\centering
\LARGE{$NFA \equiv DFA$}
		\end{frame}
		
	\section{Transducteurs finis}
	
		\begin{frame}
			\frametitle{Transducteurs finis}
\framesubtitle{}

\begin{itemize}
	\item Automate augmenté d'un mécanisme de sortie
	\item Transforme des mots de $\Sigma^*$ en des mots de $\Delta^*$
	\item Réalise une transduction $\subseteq \Sigma^* \times \Delta^*$
\end{itemize}

\begin{block}{Définition}
	Un transducteur fini de $\Sigma$ à $\Delta$ est une paire $\mathscr{T} = (\mathscr{A}, \Omega)$ où
	
	\begin{itemize}
		\item $\Delta$ est l'alphabet de sortie,
		\item $\Omega : \delta \to \Delta$ est le morphisme de sortie.

	\end{itemize}
\end{block}

\begin{itemize}
	\item Non déterministe (NFT) si $\mathscr{A}$ est non déterministe
	\item Déterministe (DFT) si $\mathscr{A}$ est déterministe
\end{itemize}


		\end{frame}
		
		\begin{frame}
			\frametitle{Transducteurs finis}
\framesubtitle{}

Déterministe (DFT)
	\begin{figure}
	\centering
	\begin{tikzpicture}[%
	>=stealth,
	shorten >=1pt,
	node distance=3cm,
	on grid,
	auto,
	state/.append style={fill=state, minimum size=1.5em},
	thick,
	]
	
	\node[state,initial,initial text={}, accepting]         (A)                    {$q_0$};
	\node[state]                       (B) [right of=A] {$q_1$};
	
	\path[->]
	(A) edge [bend left]  node {$a/a$} (B)
		edge [loop above] node {$b/\varepsilon$} ()
	(B) edge [bend left]  node {$a/a$} (A)
		edge [loop above] node {$b/\varepsilon$} ();
	
	\end{tikzpicture}
	
	\label{DFT}
\end{figure}
	
\begin{exampleblock}{Exemple}
	Retire tous les symboles $b$ d'une entrée sur $\{a,b\}^*$.	
\end{exampleblock}


\begin{itemize}
	\item Un seul chemin compatible avec chaque entrée!
	
	\begin{quotation}
		$\rightarrow$ Une seule sortie possible pour une entrée
	\end{quotation}
\end{itemize}

		\end{frame}
		
		\begin{frame}
			\frametitle{Transducteurs finis}
\framesubtitle{}

Non-déterministe (NFT)
	\begin{figure}
	\centering
	\begin{tikzpicture}[%
	>=stealth,
	shorten >=1pt,
	node distance=2.5cm,
	on grid,
	auto,
	state/.append style={fill=state, minimum size=1.5em},
	thick,
	]
	
		\node[state,initial,initial text={},initial above]         (A)                    {$q_0$};
		\node[state]                                 (B) [right of=A] {$q_1$};
		\node[state,accepting]                                 (C) [right of=B] {$q_2$};
		\node[state]                                 (D) [left of=A] {$q_3$};
		\node[state,accepting]                       (E) [left of=D]       {$q_4$};
		
		\path[->]
		(A) edge              node {$a/a$} (B)
			edge              node [swap] {$a/a$} (D)
		(B) edge [loop above] node {$b/\varepsilon$} ()
			edge   			 node {$a/a$} (C)
		(D) edge [loop above] node {$b/c$} ()
			edge  node [swap] {$a/a$} (E);
	
	\end{tikzpicture}
	
	\label{NFT}
\end{figure}
	
\begin{exampleblock}{Exemple}
	$R(\mathscr{T}) : \begin{cases}
	ab^na \mapsto aa \\
	ab^na \mapsto ac^na \\
	\end{cases}$	
\end{exampleblock}


\begin{itemize}
	\item Plusieurs chemins compatibles pour une même entrée!
	
	\begin{quotation}
		$\rightarrow$ Plusieurs sorties possibles pour une même entrée
	\end{quotation}
\end{itemize}
		\end{frame}
		
		\begin{frame}
			\frametitle{Transducteurs finis}
\framesubtitle{}

\begin{columns}[t]
	\begin{column}{0.5\linewidth}
		\Large{DFT}
		\begin{itemize}
			\item \textcolor{green}{Appartenance facilement décidable}
			\item \textcolor{green}{Inclusion/équivalence décidables en PTime}
			\item \textcolor{red}{Moins expressif car réalise forcément une fonction}
		\end{itemize}
	\end{column}
	\begin{column}{0.5\linewidth}
		\Large{NFT}
		\begin{itemize}
			\item \textcolor{red}{Appartenance nécessite du backtracking}
			\item \textcolor{red}{Inclusion/équivalence indécidables}
			\item \textcolor{green}{Plus expressif}
		\end{itemize}
	\end{column}
\end{columns}

\center
\LARGE{$DFT \subsetneq NFT$}
		\end{frame}
	
	\section{Transducteurs fonctionnels}

		\begin{frame}
			\frametitle{Transducteurs fonctionnels}
\framesubtitle{}

\begin{block}{Définition}
	Une transduction $R \subseteq \Sigma^* \to \Delta^*$ est fonctionnelle si pour tout mot $u \in \Sigma^*$ il existe au plus un mot $v \in \Delta^*$ tel que $(u,v) \in R.$
\end{block}

\begin{itemize}
	\item Un transducteur est fonctionnel si il réalise une transduction fonctionnelle.
	\item Un DFT est forcément fonctionnel.
	\item Un NFT peut être fonctionnel.
\end{itemize}

\pause
\begin{block}{Théorème}
	L'inclusion et l'équivalence pour un NFT fonctionnel sont PSpace-C.
\end{block}
		\end{frame}
	
		\begin{frame}
			\frametitle{Transducteurs fonctionnels}
\framesubtitle{}

NFT fonctionnel\footnote{Exemple d'Emmanuel Filiot}
	\begin{figure}
		\centering
		\begin{tikzpicture}[%
			>=stealth,
			shorten >=1pt,
			node distance=3cm,
			on grid,
			auto,
			state/.append style={fill=state, minimum size=2em},
			thick
		]
		
		\node[state,initial below,initial text={}] (A)		     {$q_0$};
		\node[state,accepting] (B) [left of=A] {$q_2$};
		\node[state] (C) [right of=A] {$q_1$};
		
		\path[->]
		(A)    edge  node {\_$/\varepsilon$}  (B)
				edge [bend left] node {\_/\_}	  (C)
				edge [loop above] node {$a/a$}	  ()
		(B)	   edge [loop above] node {\_$/\varepsilon$} ()
		(C)	   edge [bend left] node {$a/a$}	  (A)
		edge [loop above] node {\_$/\varepsilon$}	();
		
	\end{tikzpicture}
\end{figure}
	
\begin{exampleblock}{Exemple}
	Remplace les espaces (\_) consécutifs par un simple espace et retire les espaces en fin de mot.	
\end{exampleblock}
		\end{frame}
		
		\begin{frame}
			\frametitle{Transducteurs fonctionnels}
\framesubtitle{}
	
\begin{block}{Théorème (Schutzenberger, 1975)}
	La fonctionnalité est une propriété décidable pour NFT.
\end{block}

\begin{block}{Proposition}
	Soient $u,v,w \in \Sigma^*$.
	
	Si $u$ et $v$ sont tous deux préfixes de $w$ alors $u$ et $v$ sont comparables.
\end{block}
		\end{frame}
		
		\begin{frame}
			\frametitle{Transducteurs fonctionnels}
\framesubtitle{}

Décider la fonctionnalité
	\begin{figure}
		\centering
		\begin{tikzpicture}[%
			>=stealth,
			shorten >=1pt,
			node distance=2.5cm,
			on grid,
			auto,
			state/.append style={fill=state, minimum size=2em},
			thick,
			highlight/.style={fill=green},
			edgelight/.style={draw=blue, text=blue}
		]
		
		\node[state,initial,initial text={}] (A)		     {$0$};
		
		\node[state,onslide={<2> highlight}] (B) [above right of=A] {$1$};
		\node[state,onslide={<3> highlight}] (C) [right of=B] {$2$};
		\node[state,onslide={<4> highlight}] (D) [right of=C] {$3$};
		\node[state,accepting,onslide={<5> highlight}] (E) [right of=D] {$4$};
		
		\node[state,onslide={<2> highlight}] (F) [below right of=A] {$5$};
		\node[state,onslide={<3> highlight}] (G) [right of=F] {$6$};
		\node[state,onslide={<4> highlight}] (H) [right of=G] {$7$};
		\node[state,accepting,onslide={<5> highlight}] (I) [right of=H] {$8$};
		
		\path[->]
		(A)    edge [onslide={<2> edgelight}] node {$a/d$} (B)
			   edge [onslide={<2> edgelight}] node [swap] {$a/d$}	(F)
			   
		(B)	   edge [onslide={<3> edgelight}] node {$b/c$} (C)
		(C)	   edge [onslide={<4> edgelight}] node {$c/b$}	(D)
		(D)	   edge [onslide={<5> edgelight}] node {$d/a$}	(E)
		
		(F)	   edge [onslide={<3> edgelight}] node [swap] {$b/\varepsilon$} (G)
		(G)	   edge [onslide={<4> edgelight}] node [swap] {$c/cba$} (H)
		(H)	   edge [onslide={<5> edgelight}] node [swap] {$d/\varepsilon$} (I);
		
	\end{tikzpicture}
\end{figure}
	
	\centering
	\only<2,3,4,5>{Sorties pour} %
	\only<2>{\textcolor{green}{a}bcd :}
	\only<3>{\textcolor{green}{ab}cd :}
	\only<4>{\textcolor{green}{abc}d :}
	\only<5>{\textcolor{green}{abcd} :}
	\only<2,3,4,5>{$\begin{cases}
		
		\only<2>{\textcolor{blue}{d}\\}
		\only<2>{\textcolor{blue}{d}}
		
		\only<3>{d\textcolor{blue}{c} \\}
		\only<3>{d}
	
		\only<4>{dc\textcolor{blue}{b}\\} 
		\only<4>{d\textcolor{blue}{cba}}
	
		\only<5>{dcb\textcolor{blue}{a}\\}
		\only<5>{dcba}
		
	\end{cases}$}

	\only<6>{A faire pour chaque paire de chemins réussis sur une même entrée!}

		\end{frame}
		
		\begin{frame}
			\frametitle{Transducteurs fonctionnels}
\framesubtitle{}

\begin{exampleblock}{Calcul du carré d'un transducteur}
	Pour un transducteur $\mathscr{T} = (\mathscr{A} = (\Sigma, Q, I, F, \delta), \Omega)$ de $\Sigma$ à $\Delta$.
	
	\begin{itemize}
		\item Les états de $\mathscr{T}^2$ sont l'ensemble $Q \times Q$,
		\item Pour chaque paire de transitions $p \xrightarrow[\mathscr{T}]{a/u} q$ et $r \xrightarrow[\mathscr{T}]{a/v} s$ on crée une transition $(p,r) \xrightarrow[\mathscr{T} \times \mathscr{T}]{a/(u, v)} (q,s)$
		\item Transducteur carré de $\Sigma$ à $\Delta \times \Delta$
		\item Crée potentiellement des états inutiles
		\item Complexité en $O(|Q|^2 + |\delta|^2)$
	\end{itemize}
\end{exampleblock}


		\end{frame}
		
		\begin{frame}
			\frametitle{Transducteurs fonctionnels}
\framesubtitle{}

Dans $\mathscr{T}$:
	\begin{figure}
		\centering
		\begin{tikzpicture}[%
			>=stealth,
			shorten >=1pt,
			node distance=3cm,
			on grid,
			auto,
			state/.append style={fill=state, minimum size=2em},
			thick
		]
		
		\node[state] (A)		     {$2$};
		\node[state] (B) [right of=A] {$3$};
	
		\node[state] (C) [right of=B] {$6$};
		\node[state] (D) [right of=C] {$7$};
		
		\path[->]
		(A)    edge   node {$c/b$} 		 (B)
		(C)	   edge node {$c/cba$} 		  (D);
		
	\end{tikzpicture}
\end{figure}

Dans $\mathscr{T} \times \mathscr{T}$:
	\begin{figure}
		\centering
		\begin{tikzpicture}[%
			>=stealth,
			shorten >=1pt,
			node distance=3cm,
			on grid,
			auto,
			state/.append style={fill=state, minimum size=2em},
			thick
		]
		
		\node[state] (AC)	{$2,6$};
		\node[state] (BD) [right of=AC] {$3,7$};
		
		\path[->]
			(AC)	edge  node {$c/(b, cba)$}	(BD);
		
	\end{tikzpicture}
\end{figure}
		\end{frame}
		
		\begin{frame}
			\frametitle{Transducteurs fonctionnels}
\framesubtitle{}

$\mathscr{T} \times \mathscr{T}$:
	\begin{figure}
	\centering
	\begin{tikzpicture}[%
	>=stealth,
	shorten >=1pt,
	node distance=2.5cm,
	on grid,
	auto,
	state/.append style={fill=state, minimum size=2em},
	thick,
	highlight/.style={fill=green},
	edgelight/.style={draw=blue, text=blue},
	font=\scriptsize
	]
	
	\node[state,initial,initial text={}] (A) {$0,0$};
	
	\node[state] (B) [right of=A] {$1,5$};
	\node[state] (C) [right of=B] {$2,6$};
	\node[state] (D) [right of=C] {$3,7$};
	\node[state,accepting] (E) [right of=D] {$4,8$};
	
	\path[->]
	(A)    edge node {$a/(d,d)$} (B)
	
	(B)	   edge node {$b/(c,\varepsilon)$} (C)
	(C)	   edge node {$c/(b, cba)$}	(D)
	(D)	   edge node {$d/(a,\varepsilon)$}	(E);
	
	\end{tikzpicture}
\end{figure}
		\end{frame}
		
		\begin{frame}
			\frametitle{Transducteurs fonctionnels}
\framesubtitle{}

\begin{block}{Définition (retard)}
	On définit le retard entre $u$ et $v$ comme $delay(u,v) = (u',v')$ tel que 
		\begin{itemize}
			\item $u = lu'$, 
			\item $v' = lv'$ et 
			\item $l = lcp(u,v)$
		\end{itemize}
\end{block}
\vspace{1cm}
\centering
Exemple: $delay(abbc, abc) = (bc, c)$.
		\end{frame}
		
		\begin{frame}
			\frametitle{Transducteurs fonctionnels}
\framesubtitle{}

Décider la fonctionnalité
	\begin{figure}
		\centering
		\begin{tikzpicture}[%
			>=stealth,
			shorten >=1pt,
			node distance=2.5cm,
			on grid,
			auto,
			state/.append style={fill=state, minimum size=2em, inner sep=0pt},
			thick,
			highlight/.style={fill=green},
			edgelight/.style={draw=blue, text=blue},
			every text node part/.style={align=center},
			font=\scriptsize
		]
		
		\node[state,initial,initial text={}] (A)		     {$0,0$ \\ $(\varepsilon, \varepsilon)$};
		
		\node[state,onslide={<2> highlight}] (B) [right of=A] {$1,5$ \\ $(\varepsilon, \varepsilon)$};
		\node[state,onslide={<3> highlight}] (C) [right of=B] {$2,6$ \\ $(c, \varepsilon)$};
		\node[state,onslide={<4> highlight}] (D) [right of=C] {$3,7$ \\ $(\varepsilon, a)$};
		\node[state,accepting,onslide={<5> highlight}] (E) [right of=D] {$4,8$ \\ $(\varepsilon, \varepsilon)$};
		
		\path[->]
		(A)    edge [onslide={<2> edgelight}] node {$a/(d,d)$} (B)
			   
		(B)	   edge [onslide={<3> edgelight}] node {$b/(c,\varepsilon)$} (C)
		(C)	   edge [onslide={<4> edgelight}] node {$c/(b, cba)$}	(D)
		(D)	   edge [onslide={<5> edgelight}] node {$d/(a,\varepsilon)$}	(E);
		
	\end{tikzpicture}
\end{figure}

	\centering
	\only<2,3,4,5>{Retard en } %
	\only<2>{(1,5) :}%
	\only<3>{(2,6) :}%
	\only<4>{(3,7) :}%
	\only<5>{(4,8) :}%
	\only<2>{$delay(d,d) = (\varepsilon, \varepsilon)$}
	\only<3>{$delay(c,\varepsilon) = (c, \varepsilon)$}
	\only<4>{$delay(cb,cba) = (\varepsilon, a)$}
	\only<5>{$delay(a,a) = (\varepsilon, \varepsilon)$}
	
	\only<6>{\begin{itemize}
			\item A faire pour chaque chemin réussi de $\mathscr{T}^2$!
			\item Fonctionnel si le retard calculé à chaque état final de $\mathscr{T}^2$ est $(\varepsilon, \varepsilon)$
			\begin{quotation}
				$\rightarrow$ Simple parcours de $\mathscr{T}^2 \Rightarrow$ linéaire en $|Q|^2$.
			\end{quotation}
		\end{itemize}}
		\end{frame}
		
	\section{Transducteurs sous-séquentiels}
	
		\begin{frame}
			\frametitle{Transducteurs sous-séquentiels}
\framesubtitle{}

\begin{block}{Définition}
	Un transducteur sous-séquentiel est une paire $(\mathscr{T}, \Omega_f : F \to \Delta^*)$ où
	\begin{itemize}
		\item $\mathscr{T}$ est un transducteur déterministe
		\item $\Omega_f$ associe une sortie à chaque état final, concaténée à la suite du mot produit
	\end{itemize}
\end{block}

\begin{itemize}
	\item Plus expressif que DFT
	\item Mêmes propriétés de décision que DFT
	\item Très intéressant en pratique
\end{itemize}
		\end{frame}
		
		\begin{frame}
			\frametitle{Transducteurs sous-séquentiels}
\framesubtitle{}

\begin{figure}
	\centering
	\begin{tikzpicture}[%
	>=stealth,
	shorten >=1pt,
	node distance=3cm,
	on grid,
	auto,
	state/.append style={fill=state, minimum size=1.5em},
	thick,
	]
	
	\node[state,initial,initial text={}]  (A) {$q_0$};
	\node[state]                       (B) [right of=A] {$q_1$};
	\node[state,accepting]             (C) [right of=B] {$q_2$};
	\node[state,accepting]             (D) [above right of=B] {$q_3$};
	
	\path[->]
	(A) edge  node {$a/b$} (B)
	
	(B) edge [bend left=15]  node {$a/\varepsilon$} (C)
	edge  node {$b/aac$} (D)
	
	(C) edge [bend left=15]  node {$a/\varepsilon$} (B)
	edge  node [swap] {$b/ac$} (D)
	edge  node {$a$} +(1,0)
	
	(D) edge [loop above] node {$a/a$} ();
	
	\end{tikzpicture}
\end{figure}
		\end{frame}
		
		\begin{frame}
			\frametitle{Transducteurs sous-séquentiels}
\framesubtitle{}

Déterminisation : étendre la construction des sous-ensembles

\begin{figure}
	\centering
	\begin{tikzpicture}[%
	>=stealth,
	shorten >=1pt,
	node distance=1.8cm,
	on grid,
	auto,
	state/.append style={fill=state, minimum size=1.2em, inner sep=0pt},
	thick,
	font=\scriptsize
	]
	
	\node[state,initial,initial text={}]  (A) {$0$};
	\node[state]                      (B) [right of=A] {$1$};
	\node[state]             (C) [right of=B] {$2$};
	\node[state]             (D) [below right of=A] {$3$};
	\node[state,accepting]             (E) [right of=D] {$4$};
	\node[state,accepting]             (F) [right of=C] {$5$};
	
	\path[->]
	(A) edge  node {$a/b$} (B)
		edge [bend left=50] node {$a/ba$} (C)
		edge [swap] node {$a/ba$} (D)
	
	(B) edge [bend left=15]  node {$a/\varepsilon$} (C)
	
	(C) edge [bend left=15]  node {$a/\varepsilon$} (B)
		edge  node {$b/ac$} (F)
		
	(D) edge [bend left=15]  node {$a/\varepsilon$} (E)
	
	(E) edge [bend left=15]  node {$a/\varepsilon$} (D)
	
	(F) edge [loop above] node {$a/a$} ();
	
	\end{tikzpicture}
\end{figure}


% EN CONSTRUCTION !!!!!!!!!!!

\begin{figure}
	\centering
	\begin{tikzpicture}[%
	>=stealth,
	shorten >=1pt,
	node distance=2.2cm,
	on grid,
	auto,
	state/.append style={fill=state, minimum size=1.5em, inner sep=2pt},
	thick,
	every text node part/.style={align=center},
	font=\scriptsize
	]
	
	\node[state,initial,initial text={}]  (A) {$0,\varepsilon$};
	\node[state,onslide={<1> opacity=0}]                      (B) [right of=A] {$1,\varepsilon$ \\ $2,a$ \\ $3,a$};
	\node[state,accepting,onslide={<1,2,3> opacity=0}]             (C) [right of=B] {$1,a$ \\ $2,\varepsilon$ \\ $4,a$};
	\node[state,accepting,onslide={<1,2> opacity=0}]             (D) [above right of=B] {$5,\varepsilon$};
	
	\path[->]
	(A) edge [onslide={<1> opacity=0}]  node {$a/b$} (B)
	
	(B) edge [bend left=15,onslide={<1,2,3> opacity=0}]  node {$a/\varepsilon$} (C)
		edge [onslide={<1,2> opacity=0}]  node {$b/aac$} (D)
	
	(C) edge [bend left=15,onslide={<1,2,3,4,5> opacity=0}]  node {$a/\varepsilon$} (B)
		edge [onslide={<1,2,3,4> opacity=0}]  node [swap] {$b/ac$} (D)
		edge [onslide={<1,2,3,4,5,6> opacity=0}]  node [swap] {$a$} +(1,0)
		
	(D) edge [loop above,onslide={<1,2> opacity=0}] node {$a/a$} ();
	
	\end{tikzpicture}
\end{figure}

\only<7>{}
		\end{frame}
		
		\begin{frame}
			\frametitle{Transducteurs sous-séquentiels}
\framesubtitle{}

Déterminisation : ne fonctionne pas toujours

\begin{figure}
	\centering
	\begin{tikzpicture}[%
	>=stealth,
	shorten >=1pt,
	node distance=1.8cm,
	on grid,
	auto,
	state/.append style={fill=state, minimum size=1.2em, inner sep=0pt},
	thick,
	font=\scriptsize
	]
	
	\node[state,initial above,initial text={}]  (A) {$0$};
	\node[state]                      (B) [right of=A] {$1$};
	\node[state]             (C) [left of=A] {$2$};
	\node[state,accepting]             (D) [right of=B] {$3$};
	\node[state,accepting]             (E) [left of=C] {$4$};
	
	\path[->]
	(A) edge  node {$a/b$} (B)
		edge node [swap] {$a/c$} (C)
	
	(B) edge  node {$b/b$} (D)
		edge [loop above] node {$a/b$} ()
		
	(C) edge  node {$c/c$} (E)
		edge [loop above] node [swap] {$a/c$} ();
	
	\end{tikzpicture}
\end{figure}


% EN CONSTRUCTION !!!!!!!!!!!

\begin{figure}
	\centering
	\begin{tikzpicture}[%
	>=stealth,
	shorten >=1pt,
	node distance=2.2cm,
	on grid,
	auto,
	state/.append style={fill=state, minimum size=1.5em, inner sep=2pt},
	thick,
	every text node part/.style={align=center},
	font=\scriptsize
	]
	
	\node[state,initial,initial text={}]  (A) {$0,\varepsilon$};
	\node[state,onslide={<1> opacity=0}]                      (B) [right of=A] {$1,b$ \\ $2,c$};
	\node[state,onslide={<1,2> opacity=0}]             (C) [right of=B] {$1,bb$ \\ $2,cc$};
	\node[state,onslide={<1,2,3> opacity=0}]             (D) [right of=C] {$1,bbb$ \\ $2,ccc$};
	
	\path[->]
	(A) edge [onslide={<1> opacity=0}]  node {$a/\varepsilon$} (B)
	
	(B) edge [onslide={<1,2> opacity=0}]  node {$a/\varepsilon$} (C)
	
	(C) edge [onslide={<1,2,3> opacity=0}]  node {$a/\varepsilon$} (D)
		
	(D) edge [dotted, onslide={<1,2,3,4> opacity=0}] node {$a/\varepsilon$} +(2,0);
	
	\end{tikzpicture}
\end{figure}

\only<5>{}
		\end{frame}
		
		\begin{frame}
			\frametitle{Transducteurs sous-séquentiels}
\framesubtitle{}

\begin{block}{Définition (Condition de jumelage)}
	Un transducteur vérifie la condition de jumelage si pour toute situation
	
	\begin{figure}
		\centering
		\begin{tikzpicture}[%
		>=stealth,
		shorten >=1pt,
		node distance=1.5cm,
		on grid,
		auto,
		state/.append style={fill=state, minimum size=1.5em, inner sep=2pt},
		thick,
		every text node part/.style={align=center},
		font=\scriptsize
		]
		
		\node[state,initial,initial text={}]  (A) {$q_0$};
		\node[state]             (B) [above right of=A] {$p$};
		\node[state]            (C) [below right of=A] {$q$};
				\node[state]             (E) [right of=B] {$p$};
				\node[state]            (F) [right of=C] {$q$};
		
		\path[->]
		(A) edge [decorate,decoration={snake}]  node {$u_1/v_1$} (B)
			edge [decorate,decoration={snake}]  node [swap]  {$u_1/w_1$} (C)
		
		(B) edge [decorate,decoration={snake}]  node {$u_2/v_2$} (E)
		
		(C) edge [decorate,decoration={snake}]  node {$u_2/w_2$} (F);
		
		\end{tikzpicture}
	\end{figure}
	
	On a $delay(v_1,w) = delay(v_1v_2,w_1w_2)$.
\end{block}

\begin{block}{Théorème (Choffrut, 1977)}
	Un transducteur est sous-séquentialisable si et seulement si il vérifie la condition de jumelage.
\end{block}
		\end{frame}
	
	\section{Conclusion}
	
		\begin{frame}
			\frametitle{Conclusion}
\framesubtitle{}

\begin{center}
\LARGE{$DFT \subsetneq SSNFT \subsetneq FNFT \subsetneq NFT$}
\end{center}

\begin{itemize}
	\item Implémentation JAVA des transducteurs
	\item Implémentation de fonctionnalité, sous-séquentialité, déterminisation
\end{itemize}
		\end{frame}
		
		\begin{frame}
			\frametitle{Questions}
\framesubtitle{}

\begin{center}
\LARGE{Questions}
\end{center}
		\end{frame}

\end{document}