\frametitle{Transducteurs sous-séquentiels}
\framesubtitle{}

\begin{block}{Définition (Condition de jumelage)}
	Un transducteur vérifie la condition de jumelage si pour toute situation
	
	\begin{figure}
		\centering
		\begin{tikzpicture}[%
		>=stealth,
		shorten >=1pt,
		node distance=1.5cm,
		on grid,
		auto,
		state/.append style={fill=state, minimum size=1.5em, inner sep=2pt},
		thick,
		every text node part/.style={align=center},
		font=\scriptsize
		]
		
		\node[state,initial,initial text={}]  (A) {$q_0$};
		\node[state]             (B) [above right of=A] {$p$};
		\node[state]            (C) [below right of=A] {$q$};
				\node[state]             (E) [right of=B] {$p$};
				\node[state]            (F) [right of=C] {$q$};
		
		\path[->]
		(A) edge [decorate,decoration={snake}]  node {$u_1/v_1$} (B)
			edge [decorate,decoration={snake}]  node [swap]  {$u_1/w_1$} (C)
		
		(B) edge [decorate,decoration={snake}]  node {$u_2/v_2$} (E)
		
		(C) edge [decorate,decoration={snake}]  node {$u_2/w_2$} (F);
		
		\end{tikzpicture}
	\end{figure}
	
	On a $delay(v_1,w) = delay(v_1v_2,w_1w_2)$.
\end{block}

\begin{block}{Théorème (Choffrut, 1977)}
	Un transducteur est sous-séquentialisable si et seulement si il vérifie la condition de jumelage.
\end{block}