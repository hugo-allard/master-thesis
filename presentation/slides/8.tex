\frametitle{Transducteurs fonctionnels}
\framesubtitle{}

\begin{block}{Définition}
	Une transduction $R \subseteq \Sigma^* \to \Delta^*$ est fonctionnelle si pour tout mot $u \in \Sigma^*$ il existe au plus un mot $v \in \Delta^*$ tel que $(u,v) \in R.$
\end{block}

\begin{itemize}
	\item Un transducteur est fonctionnel si il réalise une transduction fonctionnelle.
	\item Un DFT est forcément fonctionnel.
	\item Un NFT peut être fonctionnel.
\end{itemize}

\pause
\begin{block}{Théorème}
	L'inclusion et l'équivalence pour un NFT fonctionnel sont PSpace-C.
\end{block}