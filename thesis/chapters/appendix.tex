\chapter{Schéma XML}
\label{schema}

	\begin{lstlisting}[caption={Schéma XML pour le format de fichier}]
<?xml version="1.0" encoding="UTF-8"?>
<xs:schema xmlns:xs="http://www.w3.org/2001/XMLSchema">

<xs:element name="graph" type="graph"/>
<xs:element name="automaton" type="graph"/>
<xs:element name="transducer" type="graph"/>

	<xs:complexType name="graph">
		<xs:sequence>
			<xs:element name="nodes">
				<xs:complexType>
					<xs:sequence>
						<xs:element name="node" type="node" maxOccurs="unbounded" />
					</xs:sequence>
				</xs:complexType>
			</xs:element>

			<xs:element name="transitions">
				<xs:complexType>
					<xs:sequence>
						<xs:element name="transition" type="transition" maxOccurs="unbounded" />
					</xs:sequence>
				</xs:complexType>
			</xs:element>
		</xs:sequence>

		<xs:attribute name="size" type="xs:positiveInteger" use="required"/>
	</xs:complexType>

	<xs:complexType name="node">
		<xs:attribute name="id" type="xs:nonNegativeInteger" use="required"/>
		<xs:attribute name="initial" type="xs:boolean"/>
		<xs:attribute name="accepting" type="xs:boolean"/>
		<xs:attribute name="output" type="xs:string"/>
	</xs:complexType>

	<xs:complexType name="transition">
		<xs:attribute name="from" type="xs:nonNegativeInteger" use="required"/>
		<xs:attribute name="to" type="xs:nonNegativeInteger" use="required"/>
		<xs:attribute name="input" type="xs:string" use="required"/>
		<xs:attribute name="output" type="xs:string"/>
	</xs:complexType>		
</xs:schema>
	\end{lstlisting}

\chapter{Code source}
\label{code}

Contient le code source complet, un programme exécutable prenant en entrée et effectuant les tests tels que décrits au chapitre~\ref{implementation} ainsi que plusieurs exemples au format XML.

Le programme fourni se lance avec la commande \textbf{java -jar Prog.jar}, il prend éventuellement comme arguments l'emplacement du fichier d'entrée et l'emplacement du fichier de sortie.