
Les transducteurs finis (FSTs) sont une extension des automates finis qui, en plus d'accepter ou refuser un mot d'entrée dans un langage rationnel, génèrent un mot de sortie également dans un langage rationnel. Ils réalisent donc une relation binaire entre deux langages rationnels. Les transducteurs offrent un modèle plus puissant que les automates mais, alors que les automates jouissent de très bonnes propriétés de clôture et de décidabilité, ce n'est la plupart du temps pas le cas pour les transducteurs. En effet, la classe des automates non-déterministes et la classe des automates déterministes sont équivalentes, closes pour toute opération booléenne et les problèmes de vide, inclusion, équivalence et universalité sont tous décidables. Pour les transducteurs finis, on peut principalement considérer deux classes: la classe des transducteurs non-déterministes qui étendent les automates non-déterministes et la classe des transducteurs déterministes qui étendent les automates déterministes. Ces deux classes ne sont pas équivalentes et on peut facilement prouver que les transducteurs non-déterministes sont plus expressifs. Malheureusement, la classe des transducteurs non-déterministes n'est pas close pour l'intersection et le complément et de plus, les problèmes d'inclusion et d'équivalence sont indécidables.

Cependant, lorsque l'on considère certains transducteurs non-déterministes, ceux qui définissent une relation fonctionnelle, les problèmes d'inclusion et d'équivalence deviennent décidables. Ces transducteurs sont d'autant plus intéressant que la fonctionnalité est une propriété décidable en temps polynomial.

Une autre classe de transducteurs intéressante, celle des transducteurs sous-séquentiels, est une extension des transducteurs déterministes qui associe un mot de sortie à chaque état final. Ces transducteurs sont dès lors plus expressifs que les transducteurs déterministes mais gardent de bonnes propriétés de décidabilité. Il est également décidable en temps polynomial s'il existe un transducteur sous-séquentiel équivalent à un transducteur non-déterministe.

Ce mémoire propose une implémentation extensible des transducteurs finis et d'algorithmes de décision pour la fonctionnalité et la sous-séquentialité en temps polynomial, tels que proposés par Béal et al. \cite{Bea03}, ainsi que d'un algorithme générant un transducteur sous-séquentiel équivalent à un transducteur non-déterministe passé en entrée, tel que proposé par Béal et Carton \cite{Bea02}.

Le chapitre~\ref{terminologie} fixe les notations utilisées dans ce mémoire et introduit les notions importantes sur les langages formels avant de détailler les modèles d'automates et de transducteurs. Il se clôture par une comparaison des propriétés de décidabilité entre les différentes classes de transducteurs.
Le chapitre~\ref{fonctionnel} présente une méthode de décision de l'ambiguïté des automates à partir de leur carré cartésien et montre comment étendre cette technique pour caractériser les transducteurs fonctionnels.
Le chapitre~\ref{sequentiel} détaille les caractérisations des fonctions et transducteurs sous-séquentiels présentées par Choffrut~\cite{Cho77} et montre comment la technique employée au chapitre~\ref{fonctionnel} peut également être appliquée pour les caractériser.
Le chapitre~\ref{implementation} aborde les détails de l'implémentation d'un petit framework pour les transducteurs finis ainsi que des algorithmes permettant de vérifier les caractérisations présentées aux chapitres~\ref{fonctionnel} et~\ref{sequentiel} en temps polynomial.