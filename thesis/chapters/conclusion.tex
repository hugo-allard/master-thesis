Dans ce mémoire, nous avons étudiés les différentes classes de transducteurs classiques. En particulier, nous avons vu que les transducteurs fonctionnels et sous-séquentiels jouissaient de bonnes propriétés de décidabilité tout en étant plus expressifs que les DFTs. Ces classes de transducteurs sont donc particulièrement intéressantes en pratique. De plus, nous avons étudié les propriétés qui caractérisent les relations fonctionnelles et les fonctions sous-séquentielles, et nous avons vu qu'il existe donc des procédures pour décider ces propriétés.

L'étude des algorithmes de décision pour ces propriétés a mené à la création d'un petit framework pour les transducteurs. L'implémentation se veut le plus générale possible pour permettre d'étendre facilement les structures de base. Des extensions possibles sont les \emph{transducteurs avec pile à comportement visible} tels que décrits par Servais~\cite{Ser11}, les transducteurs pondérés très souvent utilisés par Mohri~\cite{Am03} ou les transducteurs finis \emph{symboliques}~\cite{AMV13}.